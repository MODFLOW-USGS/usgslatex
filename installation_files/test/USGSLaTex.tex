\documentclass[11pt,twoside,twocolumn]{usgslatex/usgsreport}
\usepackage{usgslatex/usgsfonts}
\usepackage{usgslatex/usgsgeo}
\usepackage{usgslatex/usgsreporta}
\graphicspath{{./Figures/}}

\renewcommand{\cooperator}
{Deltares}
\renewcommand{\reporttitle}
{Parallel Krylov Solver for the \textusgs\ Modular Groundwater Flow Model (MODFLOW-2005)}
\renewcommand{\coverphoto}{Cover.jpg}
\renewcommand{\reportseries}{Techniques and Methods}
\renewcommand{\reportnumber}{6-AXX}
\renewcommand{\reportyear}{2014}


\begin{document}
\makefrontcover
\makefrontmatter
\pagestyle{body}
\SECTION{Abstract}
The U.S. Geological Survey
\SECTION{Introduction}
L-scale ($\lambda_2$) is $\sigma = \sqrt{\pi}\lambda_2$.
Recently, Mr. LaTeX has summarized it.
\SECTION{Major Section}
\subsection{General Discussion}
The number of stations summarized 
\SECTION{Summary}
Insert paragraphs here.
\REFSECTION
\begin{thebibliography}{9}
\bibitem[Asquith(2006)]{AsquithGLD2006}
Asquith, W.H., 2006, L- and TL-moments of the generalized lambda 
distribution: Computational Statistics and Data Analysis, in
press.
\end{thebibliography}
\vspace*{\fill}
\clearpage
\pagestyle{backofreport}
\makebackcover
\end{document}